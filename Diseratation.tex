\documentclass[12pt]{report}
\usepackage[a4paper, margin=1in]{geometry}
\usepackage{setspace}
\usepackage{titlesec}
\usepackage{hyperref}
\usepackage{tocloft}
\usepackage{lipsum}
\usepackage{graphicx}
\usepackage[utf8]{inputenc}
\usepackage{newunicodechar}
\usepackage{float}
\usepackage{subcaption}
\usepackage{amsmath}

\titleformat{\chapter}[hang]{\normalfont\Huge\bfseries}{\thechapter}{2pc}{}
\titleformat{\section}[hang]{\normalfont\Large\bfseries}{\thesection}{1em}{}

\setlength{\parskip}{1em}
\setlength{\parindent}{0pt}

\begin{document}

% Title Page
\newgeometry{top=0.1cm, bottom=1.5cm, right=1.5cm, left=1.5cm}
\begin{titlepage}
    \doublespacing
    \centering
    \vspace*{2cm}
    \large
    \textbf{\MakeUppercase{Mapping Air Quality and Population Vulnerability in Delhi Using AQI Data and Traffic Emission Data Models (2018–22)}}

    \Large
    A thesis submitted by\\[0.3cm]
    \textbf{Mr. Ansh Upadhyay}\\[1cm]

    Submitted in partial fulfillment of the requirements for the award of the degree of\\[0.3cm]
    \textbf{MASTER OF SCIENCE IN GIS AND REMOTE SENSING}\\[1cm]
    \includegraphics[width=1in]{Images/amity.png}

    Internal Guide\\
    \textbf{Dr. Soumendu Shekhar Roy}\\[0.5cm]
    \normalsize
    \textbf{Amity Institute of Geo-informatics \& Remote Sensing (AIGIRS)}\\
    \textbf{AMITY UNIVERSITY}\\
    \textbf{NOIDA, Uttar Pradesh}
\end{titlepage}
\restoregeometry

\onehalfspacing

\chapter*{Acknoledgement}
I would like to express my deepest gratitude to my guide, Dr. Soumendu Shekhar Roy, from the Department of AIGIRS, for his invaluable guidance, support, and encouragement throughout the course of this research. His expertise and insights have been instrumental in shaping this study on “Mapping Air Quality and Population Vulnerability in Delhi Using AQI Data and Traffic Emission Data Models (2018–22)”.\\
I am also thankful to the faculty and staff of the AIGIRS department for providing the necessary resources and support to carry out this research. Their assistance in various stages of the study has been crucial to its completion.\\
Furthermore, I would like to extend my appreciation to my family and friends for their constant encouragement and understanding during this endeavor. Their unwavering support has been a source of strength and motivation for me.\\
Lastly, I am grateful to all those who directly or indirectly contributed to this research. Their cooperation and assistance have been vital in the successful completion of this study.\\
Thank you all for your contributions and support.

% Declaration Page
\chapter*{Declaration}
I, Ansh Upadhyay, student of M.Sc. Geographic Information System and Remote Sensing, hereby declare that the project titled "Mapping Air Quality and Population Vulnerability in Delhi Using AQI Data and Traffic Emission Data Models (2018–22)"which is submitted by me to Department of, Amity Institute of Geoinformatics and Remote sensing, Amity University, Noida, Uttar Pradesh, in partial fulfillment of requirement for the award of the degree of Masters of science in Geo-information systems and remote sensing, has not been previously formed the basis for the award of any degree, diploma or other similar title or recognition.

\vspace{10cm}
Name and signature of student \hfill Date\\
Place: Noida, Uttar Pradesh

\newpage

% Table of Contents
\tableofcontents
\newpage

\begin{abstract}
    This study investigates the spatio-temporal variations in air quality and population vulnerability in Delhi from 2018 to 2022, with a specific focus on nitrogen dioxide (NO2) emissions. Utilizing satellite-derived NO2 imagery, AQI datasets, and traffic-based emission models, the research maps pollution intensity and its correlation with densely populated and high-traffic zones. The year 2019 serves as a pre-COVID baseline, reflecting typical high NO2 concentrations across urban and industrial zones. A sharp reduction in emissions during 2020 coincides with nationwide COVID-19 lockdowns, highlighting the direct influence of anthropogenic activity on air quality. Subsequent years, 2021 and 2022, depict a gradual resurgence of emissions, aligning with economic reopening and mobility resumption, ultimately returning to pre-pandemic pollution levels.\\
    Additionally, traffic density and emission datasets were integrated to identify pollution hotspots, which were then overlaid with population vulnerability indices—factoring in population density, healthcare access, and socio-economic indicators—to identify high-risk zones across Delhi. The analysis reveals that regions in Central and Western Delhi remain most susceptible to health risks due to persistent NO2 exposure and dense habitation.\\
    This study demonstrates the potential of remote sensing, GIS-based modeling, and multi-year emission analysis to support evidence-based urban planning and air quality mitigation strategies. It also reinforces the urgent need for sustainable mobility, stricter industrial regulation, and targeted health interventions in Delhi’s most vulnerable zones. 
\end{abstract}

% Chapters
\chapter{Introduction}
Air pollution is one of the most critical environmental challenges faced by major cities today. Strong evidence supports the severe impacts of various air pollutants on water, soil, vegetation, deforestation, and human health. In Asian cities, the issue is escalating rapidly due to large-scale industrialization and urban development, especially in countries still undergoing economic growth, such as India. One of the most affected regions is the national capital, New Delhi.\\
Over the past few years, Delhi has been experiencing dangerously high levels of air pollution, significantly impacting the health and quality of life of its residents. Among the major contributors to this crisis is traffic emission, particularly the release of Nitrogen Dioxide (NO2) — a harmful gas predominantly produced by vehicle exhausts. This study aims to highlight the role of traffic-related NO2 emissions in aggravating Delhi’s air pollution, alongside the concentration of Particulate Matter (PM2.5).\\
The objective of this research is to map and analyze the spatial distribution of air pollution in Delhi by integrating remote sensing and Geographic Information System (GIS) techniques. Remote sensing plays a vital role in environmental monitoring, especially for tracking air quality parameters, while GIS offers robust tools for spatial analysis and visualization. These technologies help interpret complex datasets and examine relationships between pollution levels and influencing factors such as land cover, road density, urban sprawl, and traffic flow.\\

\chapter{Objectives}
\begin{enumerate}
    
    \item To analyze the spatial and temporal distribution of air pollution in Delhifrom 2018 to 2022, with a specific focus on PM2.5 and NO$2$ levels
    \item To assess the contribution of traffic emissions to air pollution, particularly analyzing the spatial variation of Nitrogen Dioxide (NO$2$) emissions across major traffic corridors and densely populated regions of Delhi.
    \item To utilize remote sensing techniques for monitoring and mapping air quality indicators, including satellite-derived datasets relevant to atmospheric pollutants.
    \item To integrate GIS-based tools and techniques for spatial analysis, data visualization, and modeling of air pollution patterns in correlation with urban parameters like land use, road density, and urban sprawl.
    \item To develop an air pollution model that combines AQI data, traffic emission data, and urban environmental variables to identify pollution hotspots and trends.
    \item To support better environmental planning and decision-making by providing a visual and analytical representation of pollution sources and their effects on different regions within Delhi.
    \item To recommend potential strategies for pollution control and mitigation, specifically targeting vehicular emission management based on spatial patterns revealed by the study.
    
\end{enumerate}

\chapter{Database}
Data for this research is collected from Central Pollution Control Board from year 2018 to 2022, Sentinel 5P data is collected from Google Earth Engine, and shapefiles such as Delhi district, roads are downloaded from Survey of India and OSM official website.

\chapter{Methodology}
\section{Study Area and Objective Definition}
\begin{enumerate}
    \item \textbf{Study Area}: Delhi’s administrative boundary, including all urban and suburban zones.
    \item \textbf{Objectives}:
    \begin{itemize}
        \item To access the spatial distribution and temporal variations of NO$2$ emission from traffic sources.
        \item To identify high risk zones by integrating emission data with population density and road network data.
    \end{itemize}
\end{enumerate}
\section{Data acquisition and Preprocessing}
\begin{enumerate}
    \item \textbf{Datasets Used}:
    \begin{itemize}
        \item Sentinel 5P NO$2$ emission data (2018-22) via google earth engine.
        \item Population data (district wise) from the census data 2011.
        \item Delhi administrative boundary shapefile.
        \item Road network shapefile from OpenStreetMap services.
    \end{itemize}
    \item \textbf{Preprocessing}:
    \begin{itemize}
        \item Conversion of NO$2$ data into raster format.
        \item Clipping all datasets of Delhi boundary.
        \item Reprojecting and resampling rasters for alignment (WGS84 UTM zone).
        \item Creating raster layers for population density and road density.
    \end{itemize}
\end{enumerate}
\section{Raster Creation And Reclassification.}
\begin{enumerate}
    \item NO$2$, road density, and population density layers were reclassified into 5 classes based on quantile distribution.
    \item Normalization of layers was done for standardization (scale 0 to 1).
\end{enumerate}
\section{Weighted Sum Overlay}
\begin{enumerate}
    \item A weighted overlay was performed using
    \begin{itemize}
        \item \textbf{NO$2$ Emission:} 50%
        \item \textbf{Road Density:} 30%
        \item \textbf{Population Density:} 20%
    \end{itemize}
    \item The resulting raster highlights zones of population vulnerability to NO$2$ emissions from traffic sources.
\end{enumerate}
\section{Risk Zones and Classifications}
\begin{enumerate}
    \item The final suitability map was classified into 5 risk zones:
    \begin{itemize}
        \item \textbf{Very Low}
        \item \textbf{Low}
        \item \textbf{Moderate}
        \item \textbf{High}
        \item \textbf{Very High}
    \end{itemize}
\end{enumerate}
\section{Visualization}
\begin{enumerate}
    \item Thematic maps were created for:
    \begin{itemize}
        \item NO$2$ emission (per year)
        \item Population Density
        \item Road Density
        \item Vulnerability (composite weighted sum)
    \end{itemize}
    \item Legends and classification symbology were standardized for clarity.
\end{enumerate}
\begin{figure}[H]
    \centering
    \includegraphics[width=0.5\linewidth]{"Images/Methodology.png"}
    \caption{Methodology}
    \label{fig:enter-label}
\end{figure}
\section{Utilization of GIS software for Visualization}
GIS software is utilized to visualize the spatial distribution of air pollution across different sub-districts of Delhi. Many kinds of different tools are also used for accurate visualization of air pollution trends all over Delhi.\\
This research utilized various tools within Arc GIS to process, analyze and Visualize air pollution data of Delhi. Here are the following tools used:
\begin{enumerate}
    \item \textbf{Data Preparation and Management: }
    \begin{itemize}
        \item \textbf{Project/Define Projection:} To ensure all the input datasets (NO$2$, road network, population, Delhi boundary) are in the same coordinate reference system (e.g., WGS 1984 UTM Zone 43N).
        \item \textbf{Clip:} Used to extract only the data falling within the Delhi administrative boundary from all datasets.
        \item \textbf{Resample:} Used to unify raster cells sizes for proper overlay and analysis.
    \end{itemize}
    \item \textbf{Vector and Raster Conversion}
    \begin{itemize}
        \item \textbf{Raster to points:} Used to extract the values of the pixels in the NO$2$ raster for year 2018 to 2022 for analysis and performing processing.
        \item \textbf{Polyline to Raster:} Converted road network polyline shapefile to raster format to model spatial road density and proximity influence on NO$2$ concentration.
        \item \textbf{Polygon to Raster:} Converted population or ward boundaries with attributes (e.g., population density) to raster for overlay with other rasters.
    \end{itemize}
    \item \textbf{Raster Data Analysis}
    \begin{itemize}
        \item \textbf{Raster Calculator:} Used to perform calculations such as averaging NO$2$ raster layers across multiple years or creating normalized indices.
        \item \textbf{Reclassify:} Converted continuous raster values into categorical risk classes (e.g., low, medium, high) for NO$2$, road density, and population.
        \item \textbf{Weighted Sum:} Integrated multiple raster layers (NO$2$ concentration, road density, population vulnerability) using assigned weights for risk modeling.
    \end{itemize}
    \item \textbf{Conversion and Extraction}
    \begin{itemize}
        \item \textbf{Raster to Points:} Converted final risk raster to point format to allow tabular analysis or spatial joins with other vector datasets.
        \item \textbf{Extract by Mask:} Alternative to Clip for raster — used to extract raster values limited to the Delhi shapefile mask.
        \item \textbf{Extract Multi Value to Points:} Extracted multiple raster values (NO$2$, population, road) at specific point locations.
    \end{itemize}
    \item \textbf{Data Integration and Spacial Join}
    \begin{itemize}
        \item \textbf{Spacial Join:} Used to combine attribute data (e.g., population info) with spatial layers like point features from raster-to-point conversion.
        \item \textbf{Join Field:} Linked tabular data such as zonal statistics with spatial data layers for integrated analysis.
    \end{itemize}
    \item \textbf{Visualization and Layout}
    \begin{itemize}
        \item \textbf{Symbology and Layer Properties:} Applied graduated colors, risk level classification breaks, and appropriate legends for final maps.
        \item \textbf{Layout View/Map Series:} Designed final map outputs with scale bar, north arrow, title, and legend for inclusion in the report and presentations.
    \end{itemize}
\end{enumerate}

\chapter{Study Area}
\begin{figure}[H]
    \centering
    \includegraphics[width=0.7\linewidth]{Images/Layouts/Administrative Divisions.png}
    \caption{Administrative Divisions of Delhi}
\end{figure}
Delhi is the capital of India, which is also known as the hub of education and
academic scene. Including metropolis and several pockets that cater specifically to
students seeking a focused study environment.\\
\begin{enumerate}
    \item \textbf{Geography}
    Delhi is situated in northern part of India, bordered by Haryana on three sides and
    Uttar Pradesh to the east. The cover an over all area of 1,484 square kilometers and lies on the banks of Yamuna River.
    \begin{itemize}
        \item \textbf{Central Delhi:} Historic area of Delhi ies here, with iconic landmarks like Red fort and India Gate offer uniqueness of heritage and education.
        \item \textbf{South Delhi:} This part of Delhi is known for its upscale neighborhood and most prestigious isntitution like IIT Delhi, AIMS. South Delhi provides a quieter and pleasant enviroment.
        \item \textbf{North Delhi:} Hub for diverse range of colleges like Delhi University. North Delhi offers a budget friendly enviroment for everyone.
        \item \textbf{East Delhi:} With upcoming universities and changing environment East Delhi provides both affordability and academics opportunities.
    \end{itemize}
    \item \textbf{Demography:}
    The areas mentioned above tend to have higher dominance in student's population. The demographics vary depending on the locations. South Delhi attracts affluent crowds, while North Delhi has large density of income groups. Delhi also shows the huge cultural diversity because of home to students from various regions and backgrounds.\\
    Delhi also has well connected transport system for Delhi metro to Delhi transportation corporation buses (DTCs).
\end{enumerate}

\chapter{Results and Discussion}
This section presents the key findings obtained from the application of GIS and analyses. The results section is structured to highlight the NO2 emission from the traffic sources and integerating with the population and roads density data. The pattern is as follows:
\section{Year Wise Spatio-Temporal Analysis of NO$_2$ Emissions (2018 to 2022)}
Delhi,the capital city of India, has long grappled with the critical air pollution levels, largely attributed to rapid urbanization, increasing vehicular density and inedequate transporattion planning. Nitrogen Dioxide is particularly significant due to its direct linkage with vehicular emissions and its harmful effects on both human health and the enviroment.\\
NO$2$ is a reddish-brown toxic gas produced primarily from the combustion of fossil fuels, especially in vehicles. In Delhi, the ever-growing number of private and commercial vehicles is a major contributor to NO2 levels. According to air quality reports and urban mobility data, vehicular traffic contributes a substantial portion of the city's NO$X$ (NO and NO$2$) emissions, especially during peak traffic hours in densely populated urban corridors.\\
To monitor and manage NO2 pollution effectively, sattelite plateforms like Sentinel-5P(TROPOMI)provide high-resolution spatial and temporal NO2 data.When combined with local GIS data—such as road density and population distribution—these datasets help in modeling population vulnerability to NO2 emissions from traffic sources.
\subsection{NO$_2$ Emission 2018}
\begin{figure}[H]
    \centering
    \includegraphics[width=0.7\linewidth]{Images/Layouts/2018 NO2.png}
    \caption{NO2 Emissions 2018}
\end{figure}
\begin{enumerate}
    \item \textbf{Spatial Pattern}
    \begin{itemize}
        \item The highest concentration of NO$_2$ is observed in the southereastern part of the Delhi, marked by deep redish-brown colors.
        \item The northwestern and southwestern zones display lower emissions, shown in light yellow.
    \end{itemize}
    \item \textbf{Urban Emission Hotspot}
    \begin{itemize}
        \item The region with darkest brown indicates areas with heaviest traffic congestion, dense urban infrastructure, and industrial activity - potentially areas like East Delhi, Anand Vihar, or Okhla industrial area.
    \end{itemize}
    \item \textbf{Gradient interpretation}
    \begin{itemize}
        \item The smooth gradient transition from low to high emission zones suggest a spatial dispersion of NO2 likely influenced by:
        \begin{itemize}
            \item Traffic density
            \item Industrial emission
            \item Wind direction and urban ventilation
        \end{itemize}
    \end{itemize}
\end{enumerate}
\subsection{NO2 Emission 2019}

\begin{figure}[H]
    \centering
    \includegraphics[width=0.7\linewidth]{Images/Layouts/2019 NO2.png}
    \caption{Nitrogen Dioxide Emissions 2019}
\end{figure}
The Year 2019 reflects typical NO$_2$ emission levels prior to any external disruption. High NO$_2$ concenteration are evident across major metropolitan and industrial regions.
In 2019, NO$_2$ levels over Delhi were consistently high, especially concentrated in the central urban areas, industrial belts in West Delhi, and regions along major road corridors like NH-48 and NH-44.
\begin{enumerate}
    \item \textbf{Possible Sources:}
    \begin{itemize}
        \item Dense vehicular traffic across Delhi's ring roads and arterial roads.
        \item Continuous operations of industrial zones in areas like Wazirpur, Okhla, and Mayapuri.
        \item Power plant activity in surrounding NCR regions (like Jhajjar or Dadri).
        \item Agricultural stubble burning influence visible during post-monsson months.
    \end{itemize}
    \item \textbf{Interpretation:}\\
    This year serves as the baseline reference, highlighting how anthropogenic emission dominate Delhi's air quality in the absence of any movement or industrial restrictions.
\end{enumerate}
\subsection{NO2 Emission 2020}
\begin{figure}[H]
    \centering
    \includegraphics[width=0.7\linewidth]{Images/Layouts/2020 NO2.png}
    \caption{Nitrogen Dioxide Emissios 2020}  
\end{figure}
\begin{enumerate}
    \item \textbf{Observation:}
    There is a significant drop in NO$_2$ emission over Delhi in 2020. The core city area that was previously dark red now appears visibly lighter, indicating improved air quality.
    \item \textbf{Key Events:}
    \begin{itemize}
        \item \textbf{March-May 2020} lockdown resulted in near zero vehicular traffic.
        \item Suspension of Industrial operations
        \item Sharp decline in construction-related dust and combustion.
    \end{itemize}
    \item \textbf{Interpretation:}
    The lockdown phase served as a natural experiment, clearly showcasing the positive environmental impact of reduced anthropogenic activity. NO$_2$, being a short-lived pollutant, showed almost immediate reductions, making this year the cleanest air scenario during the four-year period.
\end{enumerate}

\subsection{NO2 Emissions 2021}
\begin{figure}[H]
    \centering
    \includegraphics[width=0.7\linewidth]{Images/Layouts/2021 NO2.png}
    \caption{Nitrogen Dioxide Emissions 2021}
\end{figure}
\begin{enumerate}
    \item \textbf{Observation:}\\
    Emissions levels begin to rise again in 2021, especially after the seconf covid wave. Areas such as Central Delhi, Ring road and industrial corridors shoe moderate to high NO$_2$ levels compared to 2020.
    \item \textbf{Key Patterns:}
    \begin{itemize}
        \item Hybrid work continued for many sectors, slightly slowing traffic recovery.
        \item Gradual resumption of industrial and construction activity.
        \item Still fewer emission than in 2019, suggesting partial recovery.
    \end{itemize}
    \item \textbf{Interpretation:}\\
    2021 reflects a transition phase, where partial lockdowns and fear of covid continued to limit full-scale pollution. It presents an intermediate scenario between the clean air of 2020 and the pollution rebound of 2022.
\end{enumerate}
\subsection{NO2 Emissions 2022}
\begin{figure}[H]
    \centering
    \includegraphics[width=0.7\linewidth]{Images/Layouts/2022 NO2.png}
    \caption{Nitrogen Dioxide Emission 2022} 
\end{figure}
\begin{enumerate}
    \item \textbf{Observation:}\\
    NO$_2$ emissions in 2022 nearly return to pre pandemic levels, with high concentrations visible again over West Delhi, Central Districts, and major roads.
    \item \textbf{Contributing Factors:}
    \begin{itemize}
        \item Full resumption of all sectors.
        \item Increased vehicular traffic and public transport usage.
        \item Resumption of energy generation and thermal power plants.
        \item Rise in local contruction and industrial outputs.
    \end{itemize}
    \item \textbf{Interpretation:}\\
    The temporary air quality improvement witnessed in 2020 was not sustained. Emissions in 2022 indicate that no structural changes were implemented to retain cleaner air, and the city reverted to its previuos pollution levels.
\end{enumerate}
\section{Temporal Changes in NO2 Emissions over Delhi (2018-2022)}
\begin{figure}[H]
    \centering
    \includegraphics[width=0.7\linewidth]{Images/Layouts/ChangeDetc.png}
\end{figure}
This map illustrates the spatial distribution of changes in nitrogen dioxide (NO$_2$) concentrations across Delhi over a five-year period from 2018 to 2022. The map categorizes these changes into five levels:\textbf{Very Low, Low, Moderate, High, and Very High}.
\subsection{Spatial Trends:}
\begin{enumerate}
    \item Significant Increase in NO$_2$ in Eastern and Southereastern Delhi.
    \begin{itemize}
        \item The red and deep orange areas, indicating "High" to "Very High" levels of change, are concentrated in parts of East, Southeast, and Central Delhi.
        \item These regions, which may include zones like Patparganj, Okhla, Laxmi Nagar, and nearby industrial belts, experienced a notable rise in NO$_2$ emissions, likely due to high vehicular density, industrial activity, and population pressure.
    \end{itemize}
    \item Low to Very Low NO$_2$ Change in Peripheral Areas
    \begin{itemize}
        \item Yellow and light yellow zones across the western and northern fringes of Delhi (near borders of Haryana and less urbanized zones) indicate minimal change in NO$_2$ levels.
        \item These areas are less densely populated and likely have less traffic and industrial exposure.
    \end{itemize}
\end{enumerate}
\subsection{Spatial Implications:}
The map reveals the core to peripheral gradient in NO$_2$ change:
\begin{enumerate}
    \item Core urban zones show higher NO$_2$ increases, emphasizing the need for targeted pollution mitigation in high-density regions.
    \item Peripheral zones remain relatively stable, underlining the influence of urban land use and transport systems on air quality dynamics.
\end{enumerate}
\begin{figure}[H]
    \centering
    \begin{subfigure}[b]{0.45\textwidth}
      \includegraphics[width=\textwidth]{Images/Layouts/Bar.png}
      \caption{Yearly Average of Nitrogen Dioxide Emissions (2018-22)}
      \label{fig:bar graph}
    \end{subfigure}
    \hfill
    \begin{subfigure}[b]{0.45\textwidth}
      \includegraphics[width=\textwidth]{Images/Layouts/Line.png}
      \caption{Temporal Changes of Nitrogen Dioxide Emissions (2018-22)}
      \label{fig:line graph}
    \end{subfigure}
    \caption{Nitrogen Dioxide changes over a period of Time (2018-22)}
    \label{fig:sidebyside}
  \end{figure}
  Between 2018 and 2022, average NO$_2$ emission levels showed a noticeable decline followed by a partial rebound. In 2018, emissions were at their peak, but by 2020, there was a significant reduction—likely due to decreased human activity and industrial operations. This decline was followed by a rise in 2021, suggesting a return to regular activities. However, the levels slightly decreased again in 2022, indicating a possible stabilization or the effect of improved emission control measures. Overall, the trend reflects a dip during the pandemic years and a gradual return to pre-pandemic levels, though not reaching the initial high of 2018.

\section{Roads Network Analysis of Delhi}
\begin{figure}[H]
    \centering
    \includegraphics[width=0.7\linewidth]{Images/Layouts/Roads.png}
    \caption{Roads Network of Delhi 2022}
\end{figure}
The road network of Delhi in 2022 reflects a complex and well-distributed transportation infrastructure that supports one of India's most densely populated metropolitan areas. As illustrated in the map layout, the road network is denser in the central and eastern parts of Delhi, particularly around major urban centers and commercial hubs. The peripheral and rural districts in the north-west and south-west regions show relatively sparse road development, indicating a less urbanized environment. The capital's road infrastructure is characterized by a combination of wide arterial roads, ring roads, expressways, and an intricate web of local streets that interconnect different districts.\\
Delhi’s road system is crucial in sustaining the city's mobility, commerce, and logistics. It comprises major national highways such as NH-48 and NH-44 that connect the city to neighboring states, along with the Inner Ring Road, Outer Ring Road, and other radial routes designed to ease intra-city travel. The map's scale bar and north arrow help establish spatial orientation, while the legend clearly distinguishes between the road networks and district boundaries, aiding in visual analysis. The road infrastructure has expanded over the years to accommodate rapid urbanization, although it still faces challenges like congestion, poor maintenance in some zones, and increasing vehicular pressure. Nevertheless, Delhi's road network plays a vital role in the city's economic and social dynamics, making it a core component of urban planning and development strategies.

\section{Urban areas and Settlements}
\begin{figure}[H]
    \centering
    \includegraphics[width=0.7\linewidth]{Images/Layouts/Urban Class.png}
\end{figure}
The map displays the urban extent of Delhi, highlighting built-up and urban settlement areas in red, while the non-urban or undeveloped spaces are left blank (white). The base year for this classification is 2022, and the data clearly represents the distribution of urban development across the National Capital Territory (NCT) of Delhi.
\subsection{Key Observation}
\begin{enumerate}
    \item Widespread Urbanization:\\
    A significant portion of Delhi, particularly in the central, southern, and eastern zones, is classified as urban. This reflects high population density, infrastructure development, and concentrated human activities.
    \item Dense Urban Core:\\
    The central region, encompassing areas like Connaught Place, Karol Bagh, Chandni Chowk, and Daryaganj, shows a continuous and dense urban fabric. This is historically the commercial and administrative heart of Delhi.
    \item Southern and Southereastern Urban Expansion:\\
    Extensive urban spread is visible in the South Delhi areas including Saket, Malviya Nagar, Hauz Khas, Lajpat Nagar, and towards the Okhla industrial belt. These areas combine residential, commercial, and institutional land use, making them highly urbanized.
    \item Eastern Urban Concenteration:\\
    High-density urban pockets in the Trans-Yamuna region (e.g., Shahdara, Laxmi Nagar, Mayur Vihar) indicate rapid urban growth in East Delhi. This region has seen major development post-1990s due to increasing housing demand.
    \item Patchy but Expanding Development in the West and North West:\\
    Rohini, Pitampura, Janakpuri, and Dwarka are major planned residential hubs in the west and north-west, surrounded by a mix of semi-urban and developing areas.
    \item Peripheral Semi-Urban Areas:\\
    While core and mid-regions of Delhi are densely urban, the north-east, south-west, and some southern borders still maintain relatively sparse urban coverage. These could include areas like Najafgarh, Narela, Bawana, and Chhatarpur, where urbanization is ongoing but interspersed with agricultural or vacant land.
    \item Implications for Urban Planning and Air Pollution:\\
    \begin{itemize}
        \item The extensive red patches reflect built-up density, a factor that correlates with vehicular congestion, reduced green cover, and higher NO$_2$ levels—critical inputs in your air pollution and emission risk assessment.
        \item This map validates the inclusion of land use/land cover (LULC) in your weighted overlay, especially in identifying zones with high human exposure to traffic-generated NO$_2$
    \end{itemize}
\end{enumerate}

\section{Population Vulnerability towards Nitrogen Dioxide Emission through Traffic Sources by Using Weighted Sum Method}
\subsection{What is Weighted Sum Method}
The Weighted Sum Method (WSM) is a widely adopted multi-criteria decision-making (MCDM) technique employed in GIS-based spatial analysis to assess vulnerability. In the context of population vulnerability assessment, this method enables the integration of multiple thematic layers—each representing different contributing factors—into a single composite index that reflects the spatial variation in vulnerability levels.\\
The fundamental principle of the weighted sum method lies in assigning relative importance (weights) to various contributing indicators based on their influence on population vulnerability. These indicators may include population density, proximity to roads, healthcare accessibility, literacy rate, housing conditions, age demographics, and socio-economic parameters. Each input layer is standardized or normalized to a common scale (typically 0–1 or 1–5) to ensure comparability. Normalization is essential as the raw data of each indicator may vary in units, scale, and distribution.\\
Once normalized, each layer is multiplied by its assigned weight—determined through expert judgment, literature review, or statistical approaches such as AHP (Analytical Hierarchy Process). The weighted raster layers are then aggregated through cell-wise summation to produce a composite output raster. This output represents the spatial distribution of vulnerability, where higher values indicate areas with greater susceptibility due to population characteristics and infrastructural disadvantages.\\
Mathematically, the weighted sum is expressed as:\\
\[
V = \sum_{i=1}^{n} w_i \cdot x_i
\]
Where:\\
\begin{itemize}
    \item $V$ is the total vulnerability score,
    \item $x_i$ is the normalized value of the $i^{\text{th}}$ criterion,
    \item $w_i$ is the weight assigned to the $i^{\text{th}}$ criterion,
    \item $n$ is the total number of criteria.
\end{itemize}
This method offers a flexible and transparent approach to combine multiple datasets into a unified vulnerability index. Moreover, it facilitates sensitivity analysis and scenario development by allowing adjustment of weights based on evolving priorities or policy focus. In the case of Delhi’s population vulnerability, the weighted sum approach effectively synthesizes demographic, infrastructural, and socio-economic layers to delineate hotspots requiring urgent policy intervention and resource allocation.\\
The final vulnerability map generated through this technique aids in disaster preparedness, urban planning, and risk reduction strategies, providing decision-makers with actionable insights into which areas of the population are most at risk under adverse events.

\begin{figure}[H]
    \centering
    \includegraphics[width=0.7\linewidth]{Images/Layouts/Weighted_Sum.png}
    \caption{Population Vulnerability Towards Traffic Emissions}
    \label{fig:enter-label}
\end{figure}

\section{Interpretation:}
The map titles "The Emission Risk for Population in Delhi (2018-22)" presents a spatial risk assessment derived through a weighted overlay analysis in a Geographic Information System environment. This approach integrates multiple thematic raster layers related to vehicular emissions and population exposure, resulting in a composite map that illustrates the varying levels of risk faced by the residents due to traffic-induced air pollution over a five-year period.
\subsection{Methodological Insight:}
The weighted overlay technique follows a multi-criteria decision analysis (MCDA) framework where contributing factors are normalized, classified, and assigned weights based on their relative influence on overall risk. For this particualar map factors include:
\begin{enumerate}
    \item Traffic volume or density maps
    \item Emission concentration data (eg: NOx data and PM2.5)
    \item Popultion Distribution or census grids
    \item Proximity to roads network
    \item Land use patterns
\end{enumerate}
Each raster layer is standardized into a common suitability scale and overlaid with respective weights to generate a weighted sum output.

\subsection{Classification and Symbology}
The risk is classified into five categories based on the final weighted values:
\begin{enumerate}
    \item Very low Risk
    \item Low Risk
    \item Moderate Risk
    \item High Risk
    \item Very High Risk
\end{enumerate}
This graduation enables an intrutive visualization of the zones most and least affected ny traffic- related emission exposure. The classification aids in spatial decision-making and enviromental eisk mitigation strategies.

\subsection{Spatial Pattern and Observations}
\subsubsection{High and Very High Risk Zones}
These zones are predominantly concentrated in:
\begin{enumerate}
    \item Central Delhi, inlcuding parts of Connaught Place, Daryaganj, and Karol Bagh.
    \item South-Central Delhi, notably Lajpat Nagar, Graeter Kailash and South Enxtension.
    \item Southeast and East Delhi, such as Okhla, Kalkaji, Badarpur, and Patparganj.
\end{enumerate}
These areas exhibit high population densities and lie in close proximity to major traffic corridors, ring roads, and intersections with high vehicular activity. Furthermore, these zones likely experience commercial congestion, limited green cover, and poor dispersion of vehicular pollutants, intensifying exposure risks.

\subsubsection{Moderate Risk Zones}
Moderate-risk areas act as transitional belts between high-risk cores and the peripheral low-density zones. These may include:
\begin{enumerate}
    \item Residential suburbs near arterial roads
    \item Urban Villages experiencing gradual densification
    \item Mixed land use zones with moderate traffic inflow
\end{enumerate}
These areas likely face moderate exposure due to sub-arterial traffic, indirect pollution flow, or population clusters on the fringe of core urban zones.

\subsubsection{Low and Very Low Risk Zones}
These regions are mostly situated in:
\begin{enumerate}
    \item \textbf{Outer Delhi}, especially the western (e.g., Najafgarh), southwestern (e.g., Chhawla), and northwestern peripheries.
    \item Northern rural belts and bordering regions near Haryana and Uttar Pradesh.
\end{enumerate}
These zones are typically less urbanized, have lower population densities, and limited vehicular load. The presence of open areas, agricultural land, and relatively better air dispersion contributes to their categorization as low-risk zones.

\subsection{Spatial Gradient and Urban Influence}
A discernible core-to-periphery gradient is evident in the map, with emission risk intensity gradually decreasing as one moves from central Delhi toward the outer edges. This pattern reflects:
\begin{enumerate}
    \item The urban sprawl effect
    \item The spatial distribution of vehicular networks
    \item Centralization of commercial and industrial hubs
    \item Higher population vulnerability in compact urban zones
\end{enumerate}
Such spatial gradients are critical for identifying priority zones for urban air quaility management and targeted health interventions.

\chapter{Discussion}
This study presents a spatially integrated assessment of traffic-related Nitrogen Dioxide (NO$_2$) emission risk to the population in Delhi over the period 2018–2022, using a GIS-based weighted overlay analysis. By integrating multiple geospatial datasets—namely traffic density, NO$_2$ emissions, land use/land cover, and population density—a comprehensive risk map was generated to identify areas where human exposure to traffic-originated air pollution is most significant.\\
The four thematic layers used in this analysis were standardized and assigned weights based on their relative importance:
\begin{enumerate}
    \item Traffic Density and NO$_2$ Emissions layers were given higher weights due to their direct association with vehicular pollution sources.
    \item Population Density and LULC layers helped define the level of exposure and vulnerability among residents living near emission hotspots.
\end{enumerate}
The resulting composite risk map classifies Delhi into five risk zones: Very Low, Low, Moderate, High, and Very High, offering an insightful view of spatial disparities in NO$_2$ emission exposure across the National Capital Territory.

\subsection{Urban Hotspot and NO$_2$ Concenteration}
The analysis highlights that central, south-central, and parts of eastern Delhi show consistently high to very high risk levels. These areas include regions such as Connaught Place, South Extension, Lajpat Nagar, Kalkaji, and Okhla, which are known for:
\begin{enumerate}
    \item High vehicular load and congested road networks,
    \item Dense urban development with limited green cover,
    \item Commercial and institutional zones with heavy commuter traffic
    \item High population density and mixed land-use patterns. 
\end{enumerate}
These characteristics make such zones converging points of emission, exposure, and vulnerability, explaining their classification under higher risk levels.

\subsection{Traditional Zones with Moderate Risks}
Moderate risk areas tend to be situated around the urban core and represent transitional zones such as urban villages and growing residential sectors. Though these regions may not yet have extreme traffic volumes, they are undergoing rapid urbanization. As such, they reflect a rising trend of risk due to expanding road networks and increasing vehicle dependency.

\subsection{Peripheral Zones with Low Risks}
The outer parts of Delhi, especially in the south-west, north-west, and far eastern regions, mostly show low to very low NO$_2$ risk levels. These areas typically have:
\begin{enumerate}
    \item Lower population and traffic density
    \item Semi-urban or rural land use pattern
    \item More open space and vegetation that help dissipate emissions.
\end{enumerate}
This indicates that land use planning and population density play a pivotal role in mitigating localized emission risks.

\subsection{Role of Land Use and Exposure}
The Land Use Land Cover (LULC) layer provided essential context to differentiate risk exposure levels. Zones with residential, commercial, or industrial classifications showed higher risk where combined with heavy traffic and NO$_2$ concentrations. In contrast, open lands and parks, particularly in outer Delhi or around the Delhi Ridge, contributed to lowering exposure levels even when emissions were present nearby.

\subsection{Insights from the Weighted Overlay Method}
The use of weighted overlay analysis in this study proved to be highly effective in integrating diverse but interrelated datasets into a singular, interpretable output. The method allowed for:
\begin{enumerate}
    \item Flexibility in assigning influence to each factor,
    \item Transparent aggregation of environmental and demographic risks,
    \item Spatial visualization of complex pollution dynamics across Delhi.
\end{enumerate}
This approach offers a data-driven tool for urban planners and environmental health agencies to prioritize areas for emission control, traffic regulation, and urban greening efforts.

\subsection{Nitrogen Dioxide as a Focus Pollutant}
By focusing specifically on NO$_2$ emissions, the study emphasizes a critical pollutant directly linked to vehicular activity. NO$_2$ is not only a key indicator of traffic pollution but also has severe health effects, particularly respiratory diseases, reduced lung function, and increased vulnerability to infections. The spatial risk zones identified in this study align with previously reported high-exposure zones from CPCB and satellite-derived tropospheric NO$_2$ studies, validating the significance of the findings.

\subsection{Limitation and Future Scope}
Despite providing valuable spatial insights, this research also has certain limitations:
\begin{enumerate}
    \item The temporal variability of NO$_2$ (e.g., seasonal and diurnal variation) was not accounted for.
    \item Meteorological influences such as wind speed, direction, and temperature inversions were not included.
    \item Only traffic-related NO$_2$ sources were considered, excluding industrial and household combustion contributions.
\end{enumerate}


\section{Topography}
\section{Wind Flow and Speed}
\section{Location of monitoring stations and industries}

\chapter{Conclusion}

\chapter{References}
% Add your bibliography entries here or use BibTeX for references.

\end{document}
